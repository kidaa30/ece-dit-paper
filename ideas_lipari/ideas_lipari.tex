\documentclass[a4paper,10pt]{article}
\usepackage[utf8]{inputenc}
\usepackage{charter}
\usepackage{listings}
\usepackage{color}
\usepackage{url}
\usepackage{amssymb}
\usepackage{amsmath}
\usepackage{hyperref}
\usepackage{graphicx}
\usepackage{float}
\usepackage{theorem}
\usepackage{algorithm}
\usepackage[noend]{algorithmic}
%\usepackage[section]{placeins}

\newtheorem{property}{Property}
\newtheorem{assumption}{Assumption}
\newtheorem{theorem}{Theorem}
\newtheorem{definition}{Definition}
\newtheorem{lemma}{Lemma}
\newtheorem{remark}{Remark}
\newtheorem{constraint}{Constraint}
\newtheorem{corollary}{Corollary}
\newtheorem{condition}{Condition}
\newcommand{\dbf}[1]{\operatorname{dbf}(#1)}

\definecolor{grey}{rgb}{0.9,0.9,0.9}

\lstset{
language=Python,
basicstyle=\footnotesize\fontfamily{pcr},
backgroundcolor=\color{grey},
numbers=left,
numberstyle=\tiny,
numbersep=5pt,
showstringspaces=false,
tabsize=2,
breaklines=true
}

\setlength{\parindent}{0pt}

\title{Removing Redundancy in the Set of Constraints given by the DBF Test for Asynchronous Periodic System with Constrained Deadlines}
\author{Thomas Chapeaux}
\date{Summer 2013}
%opening
\sloppy
\begin{document}
\maketitle

\tableofcontents

% \newpage

% \begin{abstract}

% \end{abstract}

\newpage

\section{The DBF test}

Let $\tau$ be a set of tasks $\tau_i = (O_i, C_i, D_i, T_i)$ where $D_i \leqslant T_i$.

\begin{definition}
    The \textbf{demand-bound function (DBF)}
    defined for a task set $\tau$ in a time interval $[t_1, t_2]$ and denoted $\dbf{t_1, t_2}$, is
    equal to the cumulated execution time of jobs of $\tau$ contained in the
    closed interval $[t_1, t_2]$.
\end{definition}

Mathematically,
\begin{equation}
    \dbf{t_1, t_2} = \sum_{i=1}^{n} n_i(t_1, t_2) \, C_i
\end{equation}
where $n_i(t_1, t_2)$ is the number of jobs of task $\tau_i$ whose arrival times
and deadlines are both in the closed interval $[t_1, t_2]$.\\

The values of the $n_i$ are given by
\begin{equation}
    n_i(t_1, t_2) = \max
    \left( 0,
        \left\lfloor
            \frac{t_2 - O_i - D_i}{T_i}
        \right\rfloor -
        \left\lceil
            \frac{t_1 - O_i}{T_i}
        \right\rceil + 1
    \right)
\end{equation}

We have the following property

\begin{theorem}
    \label{theo:DBFtest}
    \begin{equation*}
        \begin{array}{c}
            \tau \: \text{is feasible}  \iff
            \dbf{t_1, t_2} \leqslant t_2 - t_1 \; \forall t_1, t_2 \mid  0 \leq t_1 \leq t_2
        \end{array}
    \end{equation*}
\end{theorem}

However this theorem require to check the condition for an infinite number of intervals. As it turns out, only a small fraction of those conditions are necessary.

\section{Removing redundancy}

    \subsection{State of the art}

Based on previous results, we know that it is not necessary to check the condition of Theorem~\ref{theo:DBFtest} for intervals
\begin{itemize}
    \item where $t_2 > O_{max} + 2 \cdot H$
    \item where $t_1$ is not an arrival time or $t_2$ is not a deadline
\end{itemize}

Furthermore, if there is a periodic DIT at instant $t_d < O_{max} + H$, it is sufficient to check the condition for intervals such that $t_d \leqslant t_1 < t_2 \leqslant t_d + H$.\\

This already reduce the size of the set of conditions to test to a manageable size. However, in most cases there
is still redundancy in the set of conditions after applying those results, which can causes a loss of efficiency. Further analytical detection of redundancy is thus required.

    \subsection{Separation by DIT}

If there is another DIT $t_{d'}$ such that $t_d < t_{d'} < t_d + H$, then it is not necessary to check the condition in intervals where $t_1 < t_{d'} < t_2$.\\

This require to computation of every DIT modulo $H$.

% \nocite{*}
% \bibliographystyle{amsalpha}
% \bibliography{paper-paedf}

\end{document}
