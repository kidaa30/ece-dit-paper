\documentclass{beamer}
\mode<presentation>
{
  \usetheme{Warsaw}      % or try Darmstadt, Madrid, Warsaw, ...
  \usecolortheme{beaver} % or try albatross, beaver, crane, ...
  \usefonttheme{default}  % or try serif, structurebold, ...
  \setbeamertemplate{navigation symbols}{}
  \setbeamertemplate{caption}[numbered]
  \setbeamertemplate{footline}[frame number]
}


\usepackage[english]{babel}
\usepackage[utf8]{inputenc}
\usepackage{pgfpages}

\title {Real-Time Systems}
\subtitle {The Space of Feasible Execution Times for Asynchronous Periodic Task
Systems using Definitive Idle Times}
\author{Thomas~Chapeaux~\inst{1}~\inst{2} \and Paul~Rodriguez~\inst{1}~\inst{2} \and Laurent~George~\inst{2} \and Joël~Goossens~\inst{1}}
\institute[shortinst]{\inst{1} Université Libre de Bruxelles \and %
                      \inst{2} ECE Paris}
\date{July 2013}

\newcommand{\dbf}[1]{\operatorname{dbf}(#1)}

\begin{document}

\maketitle{}

\begin{frame}
	\tableofcontents
\end{frame}

\section{Introduction}

\begin{frame}
    \tableofcontents[currentsection]
\end{frame}

	\subsection{Real-time systems}

	\begin{frame}{Model}
  \begin{block}{Definition}
  Systems with real-time constraints. (e.g. ABS, VOD, etc)
  \end{block}
  \textbf{Model:} Set of tasks $\tau_i = (O_i, T_i, D_i, C_i)$ generating jobs $J_{i,j}$, with
      \begin{itemize}
      \item $O_i$ : arrival of the first job
      \item $T_i$ : time between two job arrivals
      \item $D_i$ : relative deadline
      \item $C_i$ : execution time
    \end{itemize}

  \begin{columns}[c] % the "c" option specifies center vertical alignment
  \column{.4\textwidth}
\begin{center}
\begin{tabular}{|r|c|c|c|c|}
 \hline
  & $O_i$ & $T_i$ & $D_i$ & $C_i$ \\
 \hline
 $\tau_1$ & 1 & 2 & 6 & 10\\
 \hline
 $\tau_2$ & 0 & 3 & 5 & 5\\
 \hline
\end{tabular}
\end{center}

  \column{.6\textwidth} % column designated by a command
\begin{figure}[h]
\includegraphics[width=\textwidth]{figs/RTsystem_example.png}
\caption{A task system}
\label{fig:llf}
\end{figure}
  \end{columns}
\end{frame}

    \subsection{Feasibility}

	\begin{frame}{Feasibility}
		\begin{block}{Feasibility}
        A task set is said to be \textbf{feasible} if there exists a schedule such that every job in the task set completes before its deadline is reached.
        \end{block}

        \begin{block}{Feasibility}
        A task set is said to be \textbf{schedulable by a given scheduling policy} if every job in the task set completes before its deadline is reached when using this scheduling policy.
        \end{block}

        In the uniprocessor case, the EDF (Earliest Deadline First) scheduling policy is optimal, any feasible task set is EDF-schedulable.
	\end{frame}

    \subsection{C-space}

    \begin{frame}{Definition of the C-space}
        \begin{block}{Definition}
            The \textbf{C-space} of a task system $\tau$ is a region of $n$ dimensions (where each dimension denotes the possible $C_i$ of a task of $\tau$) such that for any vector $C = \{ C_1, \cdots, C_{n}\}$ in it, $\tau$ is feasible.
        \end{block}

        \begin{itemize}
            \item A region of $\mathbb{N}_0^n$ such as the C-space is described by a set of parametric linear constraints
            \item An efficient description allows us to quickly  whether a system whose $O_i$, $T_i$ and $D_i$ values are known will be feasible on a given platform (which gives the $C_i$).
        \end{itemize}

    \end{frame}

	\section{Related works}

\begin{frame}
    \tableofcontents[currentsection]
\end{frame}

    \subsection{Demand-bound function}

    \begin{frame}{Demand-bound function feasibility test}
        \begin{block}{Definition}
            The \textbf{demand-bound function (DBF)}
            defined for a task set $\tau$ in a time interval $[t_1, t_2]$, denoted $\dbf{t_1, t_2}$, is
            the maximal cumulated execution time of jobs of $\tau$ contained in the
            closed interval $[t_1, t_2]$.
        \end{block}
        $$\dbf{t_1, t_2} = \sum_{i=1}^{n} n_i(t_1, t_2) \, C_i$$
        where $n_i(t_1, t_2)$ is the number of jobs of task $\tau_i$ whose arrival times
        and deadlines are both in the closed interval $[t_1, t_2]$.

        ~\\

        The following has been shown to be a necessary and sufficient condition of feasibility (called DBF test in the following): $$\dbf{t_1, t_2} \leqslant t_2 - t_1 \; \forall t_1 \leq t_2$$

    \end{frame}

    \subsection{C-space of synchronous systems}

    \begin{frame}{Description using the DBF test}
    The DBF test gives us a set of constraints
    $$\dbf{t_1, t_2} = \sum_{i=1}^{n} n_i(t_1, t_2) \, C_i \leqslant t_2 - t_1 \; \forall t_1 \leq t_2$$

    \begin{itemize}
        \item Every constraint is linear in the $C_i$ values
        \item Every constraint is necessary
        \item Their conjunction is sufficient
        \pause
        \item But we have an infinite number of constraints!
        \item $\Rightarrow$ Removing redundancy
    \end{itemize}
    \end{frame}

    \begin{frame}{Removing redundancy for synchronous system}

    The authors of \cite{george2009characterization} obtain an efficient description of the C-space by
    \begin{itemize}
        \item Removing constraints in which $t_1 > 0$
        \item Removing constraints in which $t_2$ is not a deadline or $t_2 > H$ (where $H = LCM(T_i)$)
    \end{itemize}

    ~\\
    \pause

    They also define the notion of \textbf{definitive idle times (DIT)} and show that the first one happens at a time $0 < t_d \leqslant H$. Finally they show that all constraints where $t_2 > t_d$ are also redundant.

    ~\\
    \pause

    The remaining constraints are thus the DBF test on intervals $[0, t]$, where $t$ is a deadline and $0 < t \leqslant t_d$. The remaining redundancies are removed by an algorithm based on the simplex.

    \end{frame}

	\subsection{Our work}

	\begin{frame}{Our work}

	In this paper, we
    \begin{itemize}
        \item Introduce the notion of \textbf{first periodic DIT (FPDIT)} as a generalization of the notion of DIT
        \item Provide an algorithm to find an efficient description of the C-space for any periodic system
        \item Quantify the feasibility gain of random offsets in terms of C-space volume ratio.
    \end{itemize}

	\end{frame}

\section{First Periodic DIT}

	\subsection{Definition}

	\begin{frame}{Definition of the FPDIT}
		\begin{block}{Definition}
			The \textbf{first periodic DIT} (FPDIT) of a system is the earliest DIT occurring
			strictly after $O_{max}$.
		\end{block}

        The FPDIT is the smallest $t_d$ such as:
       \begin{equation}
            \begin{array}{l}
                \forall \tau_i \in \tau, \exists a_i \in [D_i,T_i] \; :\\
                \left\{
                    \begin{array}{l}
                        t_d > O_i \\
                        t_d - O_i \equiv a_i \; (mod \; T_i)
                        \\
                    \end{array}
                \right.
            \end{array}
            \label{eq:FPDIT}
        \end{equation}

        For given values of $a_i$, equation~(\ref{eq:FPDIT}) has one solution modulo $H$ if the following condition is respected:
        \[
            a_i + O_i \equiv a_j + O_j \; (\operatorname{mod} \; \operatorname{gcd}(T_i,
            T_j)) \; \forall i,j
        \]

	\end{frame}

	\subsection{Finding the FPDIT value}



	\begin{frame}{Existence and value}


	\end{frame}

\section{C-space of asynchronous system}

	\begin{frame}{Redundancy in asynchronous systems}
		[$t_d$, $t_d + H$] is a feasibility interval
	\end{frame}

\section{Numerical Example}

	\begin{frame}{Example}
		Table, OCDT, figure
	\end{frame}


\bibliographystyle{IEEEtran}
\bibliography{dit-paper}

\end{document}
