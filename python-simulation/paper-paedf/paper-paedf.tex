\documentclass[a4paper,10pt]{article}
\usepackage[utf8]{inputenc}
\usepackage{charter}
\usepackage{listings}
\usepackage{color}
\usepackage{url}
\usepackage{amssymb}
\usepackage{amsmath}
\usepackage{hyperref}
\usepackage{graphicx}
%\usepackage[section]{placeins}

\definecolor{grey}{rgb}{0.9,0.9,0.9}

\lstset{
language=Python,
basicstyle=\footnotesize\fontfamily{pcr},
backgroundcolor=\color{grey},
numbers=left,
numberstyle=\tiny,
numbersep=5pt,
showstringspaces=false,
tabsize=2,
breaklines=true
}

\setlength{\parindent}{0pt}

\title{Improving schedulability rate of EDF when taking into account the preemption cost in uniprocessor systems}
\author{Thomas Chapeaux}
\date{Summer 2013}
%opening
\sloppy
\begin{document}
\maketitle

\tableofcontents

\newpage

\begin{abstract}

In preemptive scheduling techniques of real-time systems, a task can be interrupted during its execution to allow another task to meet its deadline. Previous results, such as the optimality of the EDF scheduler, assume that the added cost of the preemptions (including context saving and restoration, as well as the increase in cache misses) is negligible compared to the tasks execution times. In this paper, we show that within a model taking this cost into account, the optimality of EDF is no longer guaranteed. We then propose another algorithm which strictly dominates EDF in the model, and prove that it is also optimal for implicit deadline systems.

\end{abstract}

\newpage

\section{Introduction}



\section{Model}

    \subsection{State of the Art}

        OCDT model with per-task

        Constrained deadline

        EDF is optimal

    \subsection{Preemption Cost}

        Preemption model (add parameters alpha and atomic preemption recovery period)



        Meumeu is FTP-optimal (equivalent to ExhaustiveFTP)

\section{Performance of EDF with preemption cost}

    \subsection{Non-optimality}

    \subsection{Anomalies}

\section{PA-EDF: an amelioration}

\nocite{*}
\bibliographystyle{plain}
\bibliography{paper-paedf}


\end{document}
